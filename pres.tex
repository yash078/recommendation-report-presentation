\documentclass[12pt,table]{beamer}
    \usepackage{xcolor}
    \usepackage{subcaption}
\usepackage{tabu,longtable}
\usepackage{tcolorbox}
\usepackage[edges]{forest}
\usetheme[progressbar=frametitle]{metropolis}
\usepackage{appendixnumberbeamer}

\usepackage{booktabs}
\usepackage[scale=2]{ccicons}

\usepackage{pgfplots}
\usepgfplotslibrary{dateplot}

\usepackage{xspace}
% \renewcommand*{\bibfont}{\tiny}

\title{A recommendation report on group formation strategies at Algonquin College}
\subtitle{ENL1813T: Communications I}
% \date{\today}
\date{}
\author{Yazan A. Hussain}
\institute{IAWD - Algonquin College}
% \titlegraphic{\hfill\includegraphics[height=1.5cm]{logo.pdf}}

\begin{document}

\maketitle

\begin{frame}{Table of contents}
  \setbeamertemplate{section in toc}[sections numbered]
  \tableofcontents[hideallsubsections]
\end{frame}

\section{Introduction}

\begin{frame}[fragile]{Introduction}
        \begin{alertblock}{Scope}
        A recommendation aimed at \textbf{improving group work activities and mitigating its stress} for Algonquin College students.
        \end{alertblock}
        \begin{alertblock}{Focus} \textbf{Group formation strategies} as a tool to support group work
        \end{alertblock}
        \begin{alertblock}{Challenges}
        Establishing a measurement technique, dynamics of group work. $\then$ Proposed recommendation should be viewed as a call for action.
    \end{alertblock}
\end{frame}
\section{Methodology}
\begin{frame}[fragile]{Methodology}
% \begin{enumerate}
    \item
    \begin{alertblock}{Literature and online research}
    Basic concepts and background of group work, effect of group formation, N. American schools practice
    \end{alertblock}
    \begin{alertblock}{Survey}
    The opinion of Algonquin College students was gauged through a survey
    \end{alertblock}
    \begin{alertblock}{Faculty Opinion}
    Interview with IAWD program coordinator to get a faculty perspective
\end{alertblock}
\end{frame}

\section{Results}

\begin{frame}{Benefits of Group Work \cite{lee2015, concordia}}
        \begin{alertblock}{Academic benefits}
        \small Expansion of critical thinking skills, deeper consideration of new perspectives, preparation for professional practice, and refinement of problem-solving skills.
        \end{alertblock}
        \begin{alertblock}{Social benefits}
        \small Relationship building, self-discovery, being accountable to others, and broadening one's perspective.
        \end{alertblock}
        \begin{alertblock}{Practical benefits}
        \small Distributing workload, covering larger content%
        %of content while building on previous experiences
        ,  providing a safe environment for interaction% with others without the intimidation of being the focus on an entire class.
    \end{alertblock}
\end{frame}

\begin{frame}{Challenges to Group Work}
\begin{quote}
    \dots I decided to be as positive as possible. Perhaps the maturity level of my groups would be higher [compared to high school] in university. Boy was I wrong again. Group work only seems to get worse in university, and I can safely say that the biggest source of my school stress has come from working with my peers. \cite{macleans}
\end{quote}
\end{frame}

\begin{frame}{Challenges to Group Work}
    \begin{alertblock}{Common problems}
        Free riders, lack of exposure to all aspects of the project, and difficulties related to group dynamics \cite{chapman2006}\\(can be challenging to students and instructors \cite{facultyfocus2}).
    \end{alertblock}
    
    Motivated students are more likely to fall to group work stresses \cite{lee2015}.
    
    More severe for \emph{neurodivergent} students \cite{helping}.
\end{frame}

\begin{frame}{Group Work Domains \cite{chapman2006}}
%     \setlength{\leftmargini}{.5em}
% \begin{tabu}{|p{.32\textwidth}|p{.32\textwidth}|p{.32\textwidth}|}
%      \toprule
%      Inputs & Processes & Outcomes \\ \midrule \rowcolor{algonquin}
%          \begin{itemize}
%              \item Group homogeneity
%              \item Goal commitment
%              \item Group potency
%          \end{itemize} & \begin{itemize}
%         \item Degree of participation
%         \item workload sharing
%         \item task interdependencies
%         \item supportive group behaviors
%     \end{itemize} &
%     \begin{itemize}
%         \item Perceived performance
%         \item group viability.
%     \end{itemize} \\ \bottomrule
% \end{tabu}
    \begin{alertblock}{Inputs}
    Group homogeneity, goal commitment, group potency
    \end{alertblock}
    \begin{alertblock}{Processes} Degree of participation, workload sharing, task interdependencies, and supportive group behaviors
    \end{alertblock}
    \begin{alertblock}{Outcomes} Perceived performance and group viability.
\end{alertblock}
\end{frame}

\begin{frame}{Group Formation: Why It Matters \cite{potosky2014}}

    \begin{itemize}
        \item Group work supposed to simulate the real-world
        \item Minimizing negative group work experiences
        \item Help instructors achieve desired outcomes
\end{itemize}
\end{frame}

\begin{frame}{Group Formation Strategies \cite{chapman2006, potosky2014, pociask2017, hilton2010}}
    \begin{figure}
\hspace*{-1em}
% \centering
\begin{forest}
for tree={draw,rounded corners = 8pt, align=c,grow=0, minimum width=8em, minimum height=4em, fill=algonquin, text=white, scale = .7}
[Group formation
    [Hybrid]
    [Instructor-\\assigned
        [Informed\\assignment
            [Heterogeneous\\assignment]
            [Homogeneous\\assignment]
        ]
        [Random\\assignment]
    ]
    [Student-\\selected]
]
\end{forest}
    % \caption{Different strategies for group formation \cite{chapman2006, potosky2014, pociask2017, hilton2010}.}
    % \label{fig:group_formation}
    \end{figure}
\end{frame}

    \begin{frame}{Group Formation: Student-Selected}
 \fontsize{10.5pt}{12}\selectfont
%  \setlength{\leftmargini}{0pt}

\begin{columns}
    \begin{column}[T, onlytextwidth]{.45\textwidth}
        \begin{alertblock}{Advantages}
            \begin{itemize}
                \item Positive group experiences \cite{chapman2006, hilton2010}
                \item Better group dynamics \cite{chapman2006, hilton2010}
                \item Better communication \cite{pociask2017}
                \item Higher individual outcome \cite{hilton2010}
            \end{itemize}
        \end{alertblock}
    \end{column}
            
    \begin{column}[T, onlytextwidth]{.45\textwidth}
        \begin{alertblock}{Disadvantages}
            \begin{itemize}
                \item Inaccurate reflection of the workplace \cite{chapman2006}
                \item Weaker in time management \cite{chapman2006}, less task oriented \cite{hilton2010})
                \item Student selection criteria/attitude \cite{pociask2017}
            \end{itemize}
        \end{alertblock}
    \end{column}
\end{columns}
\end{frame}
    
    \begin{frame}{Group Formation: Instructor-Assigned}
 \fontsize{10.5pt}{12}\selectfont
%  \setlength{\leftmargini}{0pt}

\begin{columns}
    \begin{column}[T, onlytextwidth]{.45\textwidth}
        \begin{alertblock}{Advantages}
            \begin{itemize}
            \item Appears fair \cite{chapman2006}
            \item More accurate reflection of the workplace \cite{chapman2006}
            \item More diverse teams \cite{pociask2017}.
            \item Students learn to work with new coworkers \cite{chapman2006}
            \end{itemize}
        \end{alertblock}
    \end{column}
            
    \begin{column}[T, onlytextwidth]{.45\textwidth}
            \begin{alertblock}{Disadvantages} \begin{itemize}
                \item Can rely entirely on chance \cite{chapman2006} (random-assignment)
                \item Slow group startup phase \cite{chapman2006}
            \end{itemize}
        \end{alertblock}
    \end{column}
\end{columns}
\end{frame}

\begin{frame}{Group Formation in Practice}
\begin{columns}
    \begin{column}[T]{0.5\textwidth}
        \begin{alertblock}{Instructor-assigned}
        \begin{itemize}
            \item Duke University
            \item John Hopkins University
            \item University of Illinois
            \item University of Lethbridge
            \item University of Waterloo
            \item Western University
        \end{itemize}
        \end{alertblock}
    \end{column}
    \begin{column}[T]{0.5\textwidth}
        \begin{alertblock}{Open recommendation}
        \begin{itemize}
            \item Carnegie Mellon University
            \item Ryerson University
            \item University of Notre Dame
        \end{itemize}
        \end{alertblock}
    \end{column}
    % \begin{column}[T]{0.3\textwidth}
        % \begin{alertblock}{Student-selected}
        % \begin{itemize}
            % \item 
        % \end{itemize}
        % \end{alertblock}
    % \end{column}
\end{columns}
\end{frame}

\begin{frame}{Students Survey}
    % \begin{columns}
    \centering
    \begin{column}[]{0.3\textwidth}
        \hspace*{-.1in}
        \begin{tikzpicture}[nodes = {font=\sffamily}, scale=.7, every node/.style={scale=0.7}]
\def\angle{110}
\def\radius{2.5}
\def\cyclelist{{"algonquin","algonquin2","algonquin3","algonquin4"}}
\newcount\cyclecount \cyclecount=-1
\newcount\ind \ind=-1
  \foreach \percent/\name in {
      50/Student,
      40/No pref.,
      10/Instructor,
    } {
      \ifx\percent\empty\else               % If \percent is empty, do nothing
        \global\advance\cyclecount by 1     % Advance cyclecount
        \global\advance\ind by 1            % Advance list index
        \ifnum3<\cyclecount                 % If cyclecount is larger than list
          \global\cyclecount=0              %   reset cyclecount and
          \global\ind=0                     %   reset list index
        \fi
        \pgfmathparse{\cyclelist[\the\ind]} % Get color from cycle list
        \edef\color{\pgfmathresult}         %   and store as \color
        % Draw angle and set labels
        \draw[fill={\color!50},draw={\color}] (0,0) -- (\angle:\radius)
          arc (\angle:\angle+\percent*3.6:\radius) -- cycle;
        \ifnum10<\percent
        \node[align=center] at (\angle+0.5*\percent*3.6:0.5*\radius) {\percent\,\%\\(\name)};
        \else
        \node[align=center] at (\angle+0.5*\percent*3.6:0.7*\radius) {\percent\,\%};
        \node[pin=90:\name]
          at (\angle+0.5*\percent*3.6:\radius) {};
         \fi
        \pgfmathparse{\angle+\percent*3.6}  % Advance angle
        \xdef\angle{\pgfmathresult}         %   and store in \angle
      \fi
        \draw[draw=algonquin] (0,0) circle (\radius);
    };
\end{tikzpicture}
        {What do you prefer when forming a group}
    \end{column}
    \hspace{-.8in}
    \begin{column}[]{0.3\textwidth}
        \def\angle{135}
\def\radius{2.5}
\def\cyclelist{{"algonquin","algonquin2","algonquin3","algonquin4"}}
\newcount\cyclecount \cyclecount=-1
\newcount\ind \ind=-1
\begin{tikzpicture}[nodes = {font=\sffamily}, scale=.7, every node/.style={scale=0.7}]
  \foreach \percent/\name in {
    55/Abilities,
    25/Personality,
    15/Friendship,
    5/Convenience,
    } {
      \ifx\percent\empty\else               % If \percent is empty, do nothing
        \global\advance\cyclecount by 1     % Advance cyclecount
        \global\advance\ind by 1            % Advance list index
        \ifnum4<\cyclecount                 % If cyclecount is larger than list
          \global\cyclecount=0              %   reset cyclecount and
          \global\ind=0                     %   reset list index
        \fi
        \pgfmathparse{\cyclelist[\the\ind]} % Get color from cycle list
        \edef\color{\pgfmathresult}         %   and store as \color
        % Draw angle and set labels
        \draw[fill={\color!50},draw={\color}] (0,0) -- (\angle:\radius)
          arc (\angle:\angle+\percent*3.6:\radius) -- cycle;
        \ifnum24<\percent
        \node[align=center] at (\angle+0.5*\percent*3.6:0.55*\radius) {\percent\,\%\\(\name)};
        \else
        \node[align=center] at (\angle+0.5*\percent*3.6:0.7*\radius) {\percent\,\%};
        \node[pin=\angle+0.5*\percent*3.6:\name]
          at (\angle+0.5*\percent*3.6:\radius) {};
         \fi
        \pgfmathparse{\angle+\percent*3.6}  % Advance angle
        \xdef\angle{\pgfmathresult}         %   and store in \angle
      \fi
    \draw[draw=algonquin] (0,0) circle (\radius);
    };
\end{tikzpicture}
        {most important factor you consider when forming a group}
    \end{column}%\\[5ex]
    \hspace{.5in}%
    \begin{column}[]{0.3\textwidth}
        \vspace*{-.1in}
        \def\angle{100}
\def\radius{2.5}
\def\cyclelist{{"algonquin","algonquin2","algonquin3","algonquin4"}}
\newcount\cyclecount \cyclecount=-1
\newcount\ind \ind=-1
\begin{tikzpicture}[nodes = {font=\sffamily}, scale=.7, every node/.style={scale=0.7}]
  \foreach \percent/\name in {
      60/Somewhat\\stressful,
      35/Not\\stressful,
      5/Very\\stressful,
    } {
      \ifx\percent\empty\else               % If \percent is empty, do nothing
        \global\advance\cyclecount by 1     % Advance cyclecount
        \global\advance\ind by 1            % Advance list index
        \ifnum3<\cyclecount                 % If cyclecount is larger than list
          \global\cyclecount=0              %   reset cyclecount and
          \global\ind=0                     %   reset list index
        \fi
        \pgfmathparse{\cyclelist[\the\ind]} % Get color from cycle list
        \edef\color{\pgfmathresult}         %   and store as \color
        % Draw angle and set labels
        \draw[fill={\color!50},draw={\color}] (0,0) -- (\angle:\radius)
          arc (\angle:\angle+\percent*3.6:\radius) -- cycle;
        \ifnum25<\percent
        \node[align=center] at (\angle+0.5*\percent*3.6:0.5*\radius) {\percent\,\%\\(\name)};
        \else
        \node[align=center] at (\angle+0.5*\percent*3.6:0.7*\radius) {\percent\,\%};
        \node[pin={[align=center]\angle+0.5*\percent*3.6:\name}]
          at (\angle+0.5*\percent*3.6:\radius) {};
         \fi
        \pgfmathparse{\angle+\percent*3.6}  % Advance angle
        \xdef\angle{\pgfmathresult}         %   and store in \angle
      \fi
    \draw[draw=algonquin] (0,0) circle (\radius);
    };
\end{tikzpicture}
        {How stressful do you feel assigning students to a random group}
    \end{column}
\end{columns} 	 	 

    \hspace*{-3em}
\def\angle{110}
\def\radius{2.5}
\def\shiftX{0}
\def\shiftY{2}
% \def\shiftB{4}
% \def\shiftC{10}
\def\cyclelist{{"algonquin","algonquin2","algonquin3","algonquin4"}}
\newcount\cyclecount \cyclecount=-1
\newcount\ind \ind=-1
\begin{tikzpicture}[nodes = {font=\sffamily}, scale=.75, every node/.style={scale=0.75}]
\begin{scope}[shift={(\shiftX,\shiftY)}]
  \foreach \percent/\name in {
      50/Student,
      40/No pref.,
      10/Instructor,
    } {
      \ifx\percent\empty\else               % If \percent is empty, do nothing
        \global\advance\cyclecount by 1     % Advance cyclecount
        \global\advance\ind by 1            % Advance list index
        \ifnum3<\cyclecount                 % If cyclecount is larger than list
          \global\cyclecount=0              %   reset cyclecount and
          \global\ind=0                     %   reset list index
        \fi
        \pgfmathparse{\cyclelist[\the\ind]} % Get color from cycle list
        \edef\color{\pgfmathresult}         %   and store as \color
        % Draw angle and set labels
        \draw[fill={\color!50},draw={\color}] (0,0) -- (\angle:\radius)
          arc (\angle:\angle+\percent*3.6:\radius) -- cycle;
        \ifnum10<\percent
        \node[align=center] at (\angle+0.5*\percent*3.6:0.5*\radius) {\percent\,\%\\\name};
        \else
        \node[align=center] at (\angle+0.5*\percent*3.6:0.7*\radius) {\percent\,\%};
        \node[pin=90:\name]
          at (\angle+0.5*\percent*3.6:\radius) {};
         \fi
        \pgfmathparse{\angle+\percent*3.6}  % Advance angle
        \xdef\angle{\pgfmathresult}         %   and store in \angle
      \fi
    };
        \draw[draw=algonquin] (0,0) circle (\radius);
        \node[align=center, text width = 5cm] at (0,-4) {What do you prefer when forming a group};
\end{scope}

\begin{scope}[shift={(\shiftX + 7.5,\shiftY)}]
\def\angle{135}
\def\radius{2.5}
\def\cyclelist{{"algonquin","algonquin2","algonquin3","algonquin4"}}
\newcount\cyclecount \cyclecount=-1
\newcount\ind \ind=-1
% \begin{tikzpicture}[nodes = {font=\sffamily}, scale=.75, every node/.style={scale=0.75}]
  \foreach \percent/\name in {
    55/Abilities,
    25/Personality,
    15/Friendship,
    5/Convenience,
    } {
      \ifx\percent\empty\else               % If \percent is empty, do nothing
        \global\advance\cyclecount by 1     % Advance cyclecount
        \global\advance\ind by 1            % Advance list index
        \ifnum4<\cyclecount                 % If cyclecount is larger than list
          \global\cyclecount=0              %   reset cyclecount and
          \global\ind=0                     %   reset list index
        \fi
        \pgfmathparse{\cyclelist[\the\ind]} % Get color from cycle list
        \edef\color{\pgfmathresult}         %   and store as \color
        % Draw angle and set labels
        \draw[fill={\color!50},draw={\color}] (0,0) -- (\angle:\radius)
          arc (\angle:\angle+\percent*3.6:\radius) -- cycle;
        \ifnum24<\percent
        \node[align=center] at (\angle+0.5*\percent*3.6:0.55*\radius) {\percent\,\%\\\name};
        \else
        \node[align=center] at (\angle+0.5*\percent*3.6:0.7*\radius) {\percent\,\%};
        \node[pin=\angle+0.5*\percent*3.6:\name]
          at (\angle+0.5*\percent*3.6:\radius) {};
         \fi
        \pgfmathparse{\angle+\percent*3.6}  % Advance angle
        \xdef\angle{\pgfmathresult}         %   and store in \angle
      \fi
    };
    \draw[draw=algonquin] (0,0) circle (\radius);
    \node[align=center, text width = 5cm] at (0,-4) {Most important factor you consider when forming a group};
\end{scope}

\begin{scope}[shift={(\shiftX + 7.5*2,\shiftY)}]
\def\angle{100}
\def\radius{2.5}
% \def\cyclelist{{"algonquin","algonquin2","algonquin3","algonquin4"}}
\newcount\cyclecount \cyclecount=-1
\newcount\ind \ind=-1
% \begin{tikzpicture}[nodes = {font=\sffamily}, scale=.75, every node/.style={scale=0.75}]
  \foreach \percent/\name in {
      60/Somewhat\\stressful,
      35/Not\\stressful,
      5/Very\\stressful,
    } {
      \ifx\percent\empty\else               % If \percent is empty, do nothing
        \global\advance\cyclecount by 1     % Advance cyclecount
        \global\advance\ind by 1            % Advance list index
        \ifnum3<\cyclecount                 % If cyclecount is larger than list
          \global\cyclecount=0              %   reset cyclecount and
          \global\ind=0                     %   reset list index
        \fi
        \pgfmathparse{\cyclelist[\the\ind]} % Get color from cycle list
        \edef\color{\pgfmathresult}         %   and store as \color
        % Draw angle and set labels
        \draw[fill={\color!50},draw={\color}] (0,0) -- (\angle:\radius)
          arc (\angle:\angle+\percent*3.6:\radius) -- cycle;
        \ifnum25<\percent
        \node[align=center] at (\angle+0.5*\percent*3.6:0.5*\radius) {\percent\,\%\\\name};
        \else
        \node[align=center] at (\angle+0.5*\percent*3.6:0.7*\radius) {\percent\,\%};
        \node[pin={[align=center]\angle+0.5*\percent*3.6:\name}]
          at (\angle+0.5*\percent*3.6:\radius) {};
         \fi
        \pgfmathparse{\angle+\percent*3.6}  % Advance angle
        \xdef\angle{\pgfmathresult}         %   and store in \angle
      \fi
    };
    \draw[draw=algonquin] (0,0) circle (\radius);
    \node[align=center, text width = 5cm] at (0,-4) {How stressful do you feel if you are assigned to a random group};
\end{scope}
\end{tikzpicture}
\end{frame}

\begin{frame}{Students Survey}
    % \begin{figure}
    % \begin{tcolorbox}[width=\textwidth, colback = algonquin1!30]
        \centering
\begin{tikzpicture}[scale = .7, every node/.style = {scale=1}]
  \begin{axis}[
    xbar,
    xmin = 0,
    y = 35pt,
    bar width= 2em,
    % y axis line style = { opacity = 0 },
    axis x line       = none,
    axis y line       = left,
    tickwidth         = 0pt,
    symbolic y coords = {Unequal contribution to work, Different expectations, Conflicts with other students, Communication issues},
    nodes near coords,
    enlarge y limits=0.25,
    y axis line style = {-, algonquin},
    tick label style = {font=\sffamily},
  ] 
  \addplot[black, fill=algonquin] coordinates { (45,Unequal contribution to work)
                         (55,Different expectations)
                         (40,Conflicts with other students)
                         (50,Communication issues)
                      };
  \end{axis}
\end{tikzpicture}

{What would be your main concern if you are assigned to a random group?}
        % \caption{Summary of survey results for Question 4.}
        % \label{fig:survey2}
    % \end{tcolorbox}
% \end{figure}
\end{frame}


\begin{frame}{Instructor Interview}
\begin{itemize}
        \item No specific policies for group formation
        \item Ideally, random assignment should be used.
        \item Advantages and disadvantages of student-selected groups include:
        \begin{itemize}
            \item {\color{green!50!black}Good students cluster together}
            \item {\color{red}Some students are left out with no group}
            \item {\color{red}Unbalanced abilities of the different groups}
        \end{itemize}
        \item Support is available for students facing group work issues.
        \item Free-riders issue is dealt with using peer evaluation
    \end{itemize}
\end{frame}

\section{Discussion}

\begin{frame}{Recommendation}
\begin{itemize}
    \item It is recommended to use a hybrid formation approach (students are given a partial role in group formation)
    \item A hybrid method is expected to harness the advantages from both formation strategies.
    \item It can also reduce group conflicts while allowing them to push for more representative groups.
\end{itemize}
\end{frame}

\begin{frame}{Next Steps}
\begin{itemize}
    \item Algonquin College should undertake a comprehensive study on the group work situation at the College.
    \item This study should examine:
    \begin{itemize}
        \item The current practices of instructors and programs
        \item The outcomes from the perspective of constituents (instructors, students, employers)
    \end{itemize}
        \item The results of such study will help in developing guidelines to improve the group work experience including group form
\end{itemize}
\end{frame}

\begin{frame}[standout]
  \Huge Questions?
\end{frame}


\setbeamerfont{bibliography item}{size=\footnotesize}
\setbeamerfont{bibliography entry author}{size=\footnotesize}
\setbeamerfont{bibliography entry title}{size=\footnotesize}
\setbeamerfont{bibliography entry location}{size=\footnotesize}
\setbeamerfont{bibliography entry note}{size=\footnotesize}
\begin{frame}[allowframebreaks]{References}
  \bibliographystyle{IEEEtran2}
  {\small\bibliography{demo}}

\end{frame}

\end{document}
