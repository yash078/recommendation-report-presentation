\def\angle{100}
\def\radius{2.5}
\def\cyclelist{{"algonquin","algonquin2","algonquin3","algonquin4"}}
\newcount\cyclecount \cyclecount=-1
\newcount\ind \ind=-1
\begin{tikzpicture}[nodes = {font=\sffamily}, scale=.7, every node/.style={scale=0.7}]
  \foreach \percent/\name in {
      60/Somewhat\\stressful,
      35/Not\\stressful,
      5/Very\\stressful,
    } {
      \ifx\percent\empty\else               % If \percent is empty, do nothing
        \global\advance\cyclecount by 1     % Advance cyclecount
        \global\advance\ind by 1            % Advance list index
        \ifnum3<\cyclecount                 % If cyclecount is larger than list
          \global\cyclecount=0              %   reset cyclecount and
          \global\ind=0                     %   reset list index
        \fi
        \pgfmathparse{\cyclelist[\the\ind]} % Get color from cycle list
        \edef\color{\pgfmathresult}         %   and store as \color
        % Draw angle and set labels
        \draw[fill={\color!50},draw={\color}] (0,0) -- (\angle:\radius)
          arc (\angle:\angle+\percent*3.6:\radius) -- cycle;
        \ifnum25<\percent
        \node[align=center] at (\angle+0.5*\percent*3.6:0.5*\radius) {\percent\,\%\\(\name)};
        \else
        \node[align=center] at (\angle+0.5*\percent*3.6:0.7*\radius) {\percent\,\%};
        \node[pin={[align=center]\angle+0.5*\percent*3.6:\name}]
          at (\angle+0.5*\percent*3.6:\radius) {};
         \fi
        \pgfmathparse{\angle+\percent*3.6}  % Advance angle
        \xdef\angle{\pgfmathresult}         %   and store in \angle
      \fi
    \draw[draw=algonquin] (0,0) circle (\radius);
    };
\end{tikzpicture}